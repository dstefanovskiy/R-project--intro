\documentclass[a4paper,11pt]{scrartcl} 
%\usepackage[warn]{mathtext}
\usepackage[T2A]{fontenc}
%\usepackage[koi8-r]{inputenc}
\usepackage[utf8]{inputenc}
\usepackage[english,russian]{babel}
\usepackage{indentfirst}%first paragraph indent
\usepackage{cmap}
\usepackage[unicode=true]{hyperref}
\usepackage{graphicx}
\usepackage{amssymb}
\usepackage{amsmath}
\usepackage{srcltx}
\usepackage{textcomp}
\usepackage{floatflt}
\usepackage{wrapfig}
\usepackage{afterpage}
\usepackage{ccaption}
\captiondelim{. }
\usepackage{xspace}

\usepackage{wasysym}
\usepackage[noae]{Sweave}
\usepackage{underscore}
\deffootnote[2.5em]{1.5em}{1em}{\textsuperscript{\thefootnotemark}}
\setkomafont{sectioning}{\bfseries}
\setkomafont{descriptionlabel}{\bfseries}


%backslash
\newcommand{\bs}{\symbol{'134}}
%degree
\newcommand{\grad}{\ensuremath{{}^{\circ}}\xspace}
%R-акроним
\newcommand{\R}{{\sffamily\bfseries R}\xspace}

\title{Modeling  in  \R \\ and how it  ?}
\author{ Stefanovskiy D }
\date{}
\begin{document}
\maketitle

%\tableofcontents


\section{Linear and Integer Programming}
\label{sec:lpsolve}
\bigskip
Let the following data is given:

1. A certain firm can carry out projects of three types.
2. Staff consists of 4 employees, each of which is necessary for work on everyone of the projects.

Concrete data is set in  next tables:

\bigskip
\begin{tabular}{|l |c| c| c|}
\hline
& Type of Project 1 &  Type of Project 2 &  Type of Project 3 \\
\hline
Employees 1 &  2 & 4  & 5\\
\hline
Employees 2 & 1 & 8 & 6   \\
\hline
Employees 3 & 7 & 4 & 5  \\
\hline
Employees 4 & 4 & 6 & 8  \\
\hline
\end{tabular}

\bigskip

This table describe how long employees  work on each project.


\bigskip
\begin{tabular}{|l |c| }
\hline
 &  Profit from   type of project  \\
\hline
Project 1 &  1000\$  \\
\hline
Project 2 &  1400\$ \\
\hline
Project 3 &  1200\$  \\
\hline
\end{tabular}

\bigskip

This table describes how many there will be profits on sale of each project 
\bigskip

\begin{tabular}{|l |c| }
\hline
 & How many hours we have  \\
\hline
Emploiees 1 &  120 \\
\hline
Emploiees 2 & 280   \\
\hline
Emploiees 3 & 240  \\
\hline
Emploiees 4 & 360  \\
\hline
\end{tabular}

\bigskip

How many working hours in our disposal

\bigskip

Problem -> Plan of Sale?

\bigskip


\begin{tabular}{|l |c| c| c| c| }
\hline
& Type of Project 1 &  Type of Project 2 &  Type of Project 3&  \\
\hline
Employees 1 & \begin{math} 2*x_1 \end{math} & \begin{math}4*x_2  \end{math} & \begin{math} 5*x_3 \end{math} & \begin{math}  <=120 \end{math} \\
\hline
Employees 2 & \begin{math} 1*x_1 \end{math} & \begin{math} 8*x_2 \end{math} & \begin{math} 6*x_3 \end{math}& \begin{math} <=280 \end{math} \\
\hline
Employees 3 & \begin{math} 7*x_1 \end{math} & \begin{math} 4*x_2  \end{math}& \begin{math} 5*x_3 \end{math}& \begin{math} <=240 \end{math} \\
\hline
Employees 4 & \begin{math}4*x_1 \end{math}  & \begin{math} 6*x_2 \end{math} & \begin{math} 8*x_3 \end{math}& \begin{math} <=360 \end{math}  \\
\hline



\end{tabular}

\bigskip
\begin{math} F(x)  = 1000*x_1 +1400 *x_2 +1200*x_3 -> max  \end{math}\\
\begin{math}x_1, \end{math} - count of project 1 type  \\  
\begin{math}x_2, \end{math} - count of project 2 type  \\
\begin{math}x_3, \end{math} - count of project 3 type  \\ 
\\
\begin{math} 2*x_1 + 4*x_2+5*x_3 <=120 \end{math}\\
\begin{math} 1*x_1 + 8*x_2+6*x_3 <=120 \end{math}\\
\begin{math} 7*x_1 + 4*x_2+5*x_3 <=120 \end{math}\\
\begin{math} 4*x_1 + 6*x_2+8*x_3 <=120 \end{math}\\
\begin{Schunk}
\begin{Sinput}
> library("lpSolve")
\end{Sinput}
\end{Schunk}
\begin{Schunk}
\begin{Sinput}
> f.obj <- c(1000, 1400, 1200)
\end{Sinput}
\begin{Soutput}
[1] 1000 1400 1200
\end{Soutput}
\begin{Sinput}
> f.con <- matrix(c(2, 4, 5, 1, 8, 6, 7, 4, 5, 4, 6, 7), nrow = 4, 
+     byrow = TRUE)
\end{Sinput}
\begin{Soutput}
     [,1] [,2] [,3]
[1,]    2    4    5
[2,]    1    8    6
[3,]    7    4    5
[4,]    4    6    7
\end{Soutput}
\begin{Sinput}
> f.dir <- c("<=", "<=", "<=", "<=")
\end{Sinput}
\begin{Soutput}
[1] "<=" "<=" "<=" "<="
\end{Soutput}
\begin{Sinput}
> f.rhs <- c(120, 280, 240, 360)
\end{Sinput}
\begin{Soutput}
[1] 120 280 240 360
\end{Soutput}
\begin{Sinput}
> lp_var <- lp("max", f.obj, f.con, f.dir, f.rhs)
\end{Sinput}
\begin{Soutput}
Success: the objective function is 49200 
\end{Soutput}
\begin{Sinput}
> lp_var
\end{Sinput}
\begin{Soutput}
Success: the objective function is 49200 
\end{Soutput}
\begin{Sinput}
> lp_var$solution
\end{Sinput}
\begin{Soutput}
[1] 24 18  0
\end{Soutput}
\end{Schunk}
Project 1 ---> 24 units

Project 2 ---> 18 units

Project 3 ---> 0 units

Our plan of sale  is ready.
\bigskip

Now we have plan of sale and whats about addvertisment company.
After investigation  and research marketing  we  have   next data: 

Coverage  of clients

\begin{tabular}{|l |c|}
\hline
 TV &  60 \% \\
\hline
Radio  & 20 \% \\
\hline
Press  & 10 \% \\
\hline
Magazines  & 10 \% \\
\hline

\end{tabular}

For TV our firm make two clips C1(old) and C2(new)

\begin{tabular}{|l |c|}
\hline
Clips  & Like  \\

\hline
C1  & 25  \%\\
\hline
C2  & 35  \%\\
\hline

\end {tabular}
\begin{equation} Important:   25+35 == 60 \end{equation}

For Radio our firm make two clips R1(old) and R2(new)

\begin{tabular}{|l |c|}
\hline
Clips  & Like  \\

\hline
R1  & 12  \%\\
\hline
R2  & 8  \%\\
\hline

\end {tabular}
\begin{equation} Important:  12+8 == 20 \end{equation}

For Press our firm make two articles PA1(old) and PA2(new)

\begin{tabular}{|l |c|}
\hline
 Articles & Like  \\

\hline
PA1  & 4  \%\\
\hline
PA2  & 6  \%\\
\hline

\end {tabular}
\begin{equation} Important:  4+6 == 10 \end{equation}

For Journals our firm make two articles JA1(old) and JA2(new)

\begin{tabular}{|l |c|}
\hline
 Articles & Д  \\

\hline
JA1  & 4  \%\\
\hline
JA2  & 6  \%\\
\hline

\end {tabular}
\begin{equation} Important:  4+6 == 10 \end{equation}

\bigskip

Was poll of experts on which basis is moved it is offered as expenses for the publication or a mention for each advertizing material should be distributed

if to use  data of this poll we may receive  next   table

\bigskip

\begin{tabular}{|l|l |c| c| c| c| c| c| c| c|}
\hline
&      &  C1 & C2  &  PA1 & PA2 & JA1 & JA2 & R1 & R2\\
\hline
20&Radio &  5 & 5  &  5 & 5 & 5 & 5 & 35 & 35\\
\hline
10&Press &  5 & 5  &  28 & 22 &  15 & 15 & 5 & 5\\
\hline
60&TV &  25 & 25  &  5 & 5 & 5 & 5 & 15 & 15\\
\hline
10&Journals &  15 & 15  &  5 & 5 & 25 & 25 & 5 & 5\\
\hline
 & & 25 &  35 & 4  &  6 & 4 & 6 & 12 & 8 \\
\hline
\end{tabular}


\bigskip


In this case the decision is trivial and corresponds 
to preferences of clients.
We can check up it, having solved a corresponding transport problem


\bigskip



\begin{tabular}{|l |c| c| c| c| c| c| c| c|c|}
\hline
      &  C1 & C2  &  PA1 & PA2 & JA1 & JA2 & R1 & R2&\\
\hline
Radio &  \begin{math} 5* x_1{}_1 \end{math}  &  \begin{math} 5* x_1{}_2 \end{math}  & \begin{math} 5* x_1{}_3 \end{math}   & \begin{math} 5* x_1{}_4 \end{math}  &  \begin{math} 5* x_1{}_5 \end{math} &  \begin{math} 5* x_1{}_6 \end{math}  &   \begin{math} 35* x_1{}_7 \end{math} &   \begin{math} 35* x_1{}_8 \end{math}&<=20\\

\hline
 Press &  \begin{math} 5* x_2{}_1 \end{math}  &  \begin{math} 5* x_2{}_2 \end{math}  & \begin{math} 28* x_2{}_3 \end{math}   & \begin{math} 22* x_2{}_4 \end{math}  &  \begin{math} 15* x_2{}_5 \end{math} &  \begin{math} 15* x_2{}_6 \end{math}  &   \begin{math} 5* x_2{}_7 \end{math} &   \begin{math} 5* x_2{}_8 \end{math} & <=10\\





\hline
 TV &  \begin{math} 25* x_3{}_1 \end{math}  &  \begin{math} 25* x_3{}_2 \end{math}  & \begin{math} 5* x_3{}_3 \end{math}   & \begin{math} 5* x_3{}_4 \end{math}  &  \begin{math} 25* x_3{}_5 \end{math} &  \begin{math} 25* x_3{}_6 \end{math}  &   \begin{math} 5* x_3{}_7 \end{math} &   \begin{math} 5* x_3{}_8 \end{math} & <=60\\




\hline
 Journals &  \begin{math} 15* x_4{}_1 \end{math}  &  \begin{math} 15* x_4{}_2 \end{math}  & \begin{math} 5* x_4{}_3 \end{math}   & \begin{math} 5* x_4{}_4 \end{math}  &  \begin{math} 25* x_4{}_5 \end{math} &  \begin{math} 25* x_4{}_6 \end{math}  &   \begin{math} 5* x_4{}_7 \end{math} &   \begin{math} 5* x_4{}_8 \end{math} &<=10 \\

\hline
\hline

 &  <=25 & <=35 & <=4  &  <=6 & <=4 & <=6 & <=12 & <=8 & \\
\hline




\end{tabular}

\bigskip

Now solve  

\begin{Schunk}
\begin{Sinput}
> f.obj <- c(5, 5, 5, 5, 5, 5, 35, 35, 5, 5, 28, 22, 15, 15, 5, 
+     5, 25, 25, 5, 5, 5, 5, 15, 15, 15, 15, 5, 5, 25, 25, 5, 5)
\end{Sinput}
\begin{Soutput}
 [1]  5  5  5  5  5  5 35 35  5  5 28 22 15 15  5  5 25 25  5  5  5  5 15 15 15
[26] 15  5  5 25 25  5  5
\end{Soutput}
\begin{Sinput}
> f.con <- matrix(c(1, 1, 1, 1, 1, 1, 1, 1, 0, 0, 0, 0, 0, 0, 0, 
+     0, 0, 0, 0, 0, 0, 0, 0, 0, 0, 0, 0, 0, 0, 0, 0, 0, 0, 0, 
+     0, 0, 0, 0, 0, 0, 1, 1, 1, 1, 1, 1, 1, 1, 0, 0, 0, 0, 0, 
+     0, 0, 0, 0, 0, 0, 0, 0, 0, 0, 0, 0, 0, 0, 0, 0, 0, 0, 0, 
+     0, 0, 0, 0, 0, 0, 0, 0, 1, 1, 1, 1, 1, 1, 1, 1, 0, 0, 0, 
+     0, 0, 0, 0, 0, 0, 0, 0, 0, 0, 0, 0, 0, 0, 0, 0, 0, 0, 0, 
+     0, 0, 0, 0, 0, 0, 0, 0, 0, 0, 1, 1, 1, 1, 1, 1, 1, 1, 1, 
+     0, 0, 0, 0, 0, 0, 0, 1, 0, 0, 0, 0, 0, 0, 0, 1, 0, 0, 0, 
+     0, 0, 0, 0, 1, 0, 0, 0, 0, 0, 0, 0, 0, 1, 0, 0, 0, 0, 0, 
+     0, 0, 1, 0, 0, 0, 0, 0, 0, 0, 1, 0, 0, 0, 0, 0, 0, 0, 1, 
+     0, 0, 0, 0, 0, 0, 0, 0, 1, 0, 0, 0, 0, 0, 0, 0, 1, 0, 0, 
+     0, 0, 0, 0, 0, 1, 0, 0, 0, 0, 0, 0, 0, 1, 0, 0, 0, 0, 0, 
+     0, 0, 0, 1, 0, 0, 0, 0, 0, 0, 0, 1, 0, 0, 0, 0, 0, 0, 0, 
+     1, 0, 0, 0, 0, 0, 0, 0, 1, 0, 0, 0, 0, 0, 0, 0, 0, 1, 0, 
+     0, 0, 0, 0, 0, 0, 1, 0, 0, 0, 0, 0, 0, 0, 1, 0, 0, 0, 0, 
+     0, 0, 0, 1, 0, 0, 0, 0, 0, 0, 0, 0, 1, 0, 0, 0, 0, 0, 0, 
+     0, 1, 0, 0, 0, 0, 0, 0, 0, 1, 0, 0, 0, 0, 0, 0, 0, 1, 0, 
+     0, 0, 0, 0, 0, 0, 0, 1, 0, 0, 0, 0, 0, 0, 0, 1, 0, 0, 0, 
+     0, 0, 0, 0, 1, 0, 0, 0, 0, 0, 0, 0, 1, 0, 0, 0, 0, 0, 0, 
+     0, 0, 1, 0, 0, 0, 0, 0, 0, 0, 1, 0, 0, 0, 0, 0, 0, 0, 1, 
+     0, 0, 0, 0, 0, 0, 0, 1), nrow = 12, byrow = TRUE)
\end{Sinput}
\begin{Soutput}
      [,1] [,2] [,3] [,4] [,5] [,6] [,7] [,8] [,9] [,10] [,11] [,12] [,13]
 [1,]    1    1    1    1    1    1    1    1    0     0     0     0     0
 [2,]    0    0    0    0    0    0    0    0    1     1     1     1     1
 [3,]    0    0    0    0    0    0    0    0    0     0     0     0     0
 [4,]    0    0    0    0    0    0    0    0    0     0     0     0     0
 [5,]    1    0    0    0    0    0    0    0    1     0     0     0     0
 [6,]    0    1    0    0    0    0    0    0    0     1     0     0     0
 [7,]    0    0    1    0    0    0    0    0    0     0     1     0     0
 [8,]    0    0    0    1    0    0    0    0    0     0     0     1     0
 [9,]    0    0    0    0    1    0    0    0    0     0     0     0     1
[10,]    0    0    0    0    0    1    0    0    0     0     0     0     0
[11,]    0    0    0    0    0    0    1    0    0     0     0     0     0
[12,]    0    0    0    0    0    0    0    1    0     0     0     0     0
      [,14] [,15] [,16] [,17] [,18] [,19] [,20] [,21] [,22] [,23] [,24] [,25]
 [1,]     0     0     0     0     0     0     0     0     0     0     0     0
 [2,]     1     1     1     0     0     0     0     0     0     0     0     0
 [3,]     0     0     0     1     1     1     1     1     1     1     1     0
 [4,]     0     0     0     0     0     0     0     0     0     0     0     1
 [5,]     0     0     0     1     0     0     0     0     0     0     0     1
 [6,]     0     0     0     0     1     0     0     0     0     0     0     0
 [7,]     0     0     0     0     0     1     0     0     0     0     0     0
 [8,]     0     0     0     0     0     0     1     0     0     0     0     0
 [9,]     0     0     0     0     0     0     0     1     0     0     0     0
[10,]     1     0     0     0     0     0     0     0     1     0     0     0
[11,]     0     1     0     0     0     0     0     0     0     1     0     0
[12,]     0     0     1     0     0     0     0     0     0     0     1     0
      [,26] [,27] [,28] [,29] [,30] [,31] [,32]
 [1,]     0     0     0     0     0     0     0
 [2,]     0     0     0     0     0     0     0
 [3,]     0     0     0     0     0     0     0
 [4,]     1     1     1     1     1     1     1
 [5,]     0     0     0     0     0     0     0
 [6,]     1     0     0     0     0     0     0
 [7,]     0     1     0     0     0     0     0
 [8,]     0     0     1     0     0     0     0
 [9,]     0     0     0     1     0     0     0
[10,]     0     0     0     0     1     0     0
[11,]     0     0     0     0     0     1     0
[12,]     0     0     0     0     0     0     1
\end{Soutput}
\begin{Sinput}
> f.dir <- c("<=", "<=", "<=", "<=", "<=", "<=", "<=", "<=", "<=", 
+     "<=", "<=", "<=")
\end{Sinput}
\begin{Soutput}
 [1] "<=" "<=" "<=" "<=" "<=" "<=" "<=" "<=" "<=" "<=" "<=" "<="
\end{Soutput}
\begin{Sinput}
> f.rhs <- c(20, 10, 60, 10, 25, 35, 4, 6, 4, 6, 12, 8)
\end{Sinput}
\begin{Soutput}
 [1] 20 10 60 10 25 35  4  6  4  6 12  8
\end{Soutput}
\begin{Sinput}
> des <- lp("max", f.obj, f.con, f.dir, f.rhs)
\end{Sinput}
\begin{Soutput}
Success: the objective function is 2694 
\end{Soutput}
\begin{Sinput}
> mm <- matrix(des$solution, , nrow = 4, byrow = TRUE)
\end{Sinput}
\begin{Soutput}
     [,1] [,2] [,3] [,4] [,5] [,6] [,7] [,8]
[1,]    0    0    0    0    0    0   12    8
[2,]    0    0    4    6    0    0    0    0
[3,]   25   35    0    0    0    0    0    0
[4,]    0    0    0    0    4    6    0    0
\end{Soutput}
\end{Schunk}
 And, it has been told above.
 
 How to be, in that case when the vector of preference of clients about advertizing products doesn't coincide with preferences about delivery systems to its clients.
 For example we remove conditions 1,2,3,4 for each product.
 It can occur when resources are estimated together, but not in pairs.
 Therefore equality of the sum of all estimations 100 will be unique restriction.
 Then if
 
 
 
 %К примеру мы снимаем условия 1,2,3,4   для каждого продукта, это может произойти когда ресурсы оцениваются  вместе, а не попарно, поэтому   единственным условием будет не превышение суммы всех оценок 100.  Тогда, если:
 
 \begin{tabular}{|l|l |c| c| c| c| c| c| c| c|}
\hline
&      &  C1 & C2  &  PA1 & PA2 & JA1 & JA2 & R1 & R2\\
\hline
20&Radio &  5 & 5  &  5 & 5 & 5 & 5 & 35 & 35\\
\hline
10&Press &  5 & 5  &  28 & 22 &  15 & 15 & 5 & 5\\
\hline
60&TV &  25 & 25  &  5 & 5 & 5 & 5 & 15 & 15\\
\hline
10&Journals &  15 & 15  &  5 & 5 & 25 & 25 & 5 & 5\\
\hline
& &  15 & 15  &  5 & 5 & 25 & 25 & 5 & 5\\
\hline
\end{tabular}

We need  chande   only vector 
f.rhs <- c(20, 10,60,10,15 , 15  ,  5 , 5 , 25 , 25 , 5 , 5)
\begin{Schunk}
\begin{Sinput}
> f.rhs <- c(20, 10, 60, 10, 15, 15, 5, 5, 25, 25, 5, 5)
\end{Sinput}
\begin{Soutput}
 [1] 20 10 60 10 15 15  5  5 25 25  5  5
\end{Soutput}
\begin{Sinput}
> des <- lp("max", f.obj, f.con, f.dir, f.rhs)
\end{Sinput}
\begin{Soutput}
Success: the objective function is 1800 
\end{Soutput}
\begin{Sinput}
> mm <- matrix(des$solution, , nrow = 4, byrow = TRUE)
\end{Sinput}
\begin{Soutput}
     [,1] [,2] [,3] [,4] [,5] [,6] [,7] [,8]
[1,]    0    0    0    0    0   10    5    5
[2,]    0    0    5    5    0    0    0    0
[3,]   15   15    0    0   15   15    0    0
[4,]    0    0    0    0   10    0    0    0
\end{Soutput}
\end{Schunk}
 \begin{tabular}{|l|l |c| c| c| c| c| c| c| c|}
\hline
&      &  C1 & C2  &  PA1 & PA2 & JA1 & JA2 & R1 & R2\\
\hline
20&Radio &  5 & 5  &  5 & 5 & 5 & 5 & 35 & 35\\
\hline
10&Press &  5 & 5  &  28 & 22 &  15 & 15 & 5 & 5\\
\hline
60&TV &  25 & 25  &  5 & 5 & 5 & 5 & 15 & 15\\
\hline
10&Journals &  15 & 15  &  5 & 5 & 25 & 25 & 5 & 5\\
\hline
& &  15 & 15  &  5 & 5 & 25 & 25 & 5 & 5\\
\hline
\end{tabular}

This decision gives the chance to make elections on a template:

Row 1 -Radio:
It is necessary to use both radio clips and speek about JA2. 

Row 2 -Press: 
It is necessary to publish both articles

Row 3 -TV In equal quantities to place both clips and also to offer a clip on the  cdrom together with magazines

По строке 4 
It is necessary to publish only the first article in magazine, and the second to discuss on radio.


Col 1 and  2 
It is necessary to show both clips on television
Col  3  и 4
Both articles it is necessary to publish


\bigskip

Now a conclusion on cells. In them the percent from a total sum of financing which can be spent for the publication of this resource in соотвествующем mass-media is specified. 
That is, it is specified approximate финасовый the plan for an example the planned 
The sum on the advertizing company is equal 10 000 \$,
Then financing volumes turn out the following:

 \begin{tabular}{|l|l |c| c| c| c| c| c| c| c|r|}
\hline
&      &  C1 & C2  &  PA1 & PA2 & JA1 & JA2 & R1 & R2 &2000 \$\\
\hline
20&Radio &  0 & 0  &  0 & 0 & 0 & 1000\$ & 500\$ & 500\$  & 1000 \$\\
\hline
10&Press &  0 & 0  &  500 \$ & 500 \$ &  0 & 0 & 0 & 0 &1000 \$ \\
\hline
60&TV &  1500 \$ & 1500 \$  &  0 & 0 & 1500 \$ & 1500 \$ & 0 & 0 &6000 \$\\
\hline
10&Journals &  0 & 0  &  0 & 0 & 1000 \$ & 0 & 0 & 0 & 1000 \$ \\
\hline
Итого & &   1500 \$&  1500 \$  &  500 \$ & 500 \$  & 2500 \$  & 2500 \$ & 500 \$  & 500 \$ & \\
\hline
\end{tabular}










\end{document}
