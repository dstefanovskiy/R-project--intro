\documentclass[a4paper,11pt]{scrartcl} 
%\usepackage[warn]{mathtext}
\usepackage[T2A]{fontenc}
%\usepackage[koi8-r]{inputenc}
\usepackage[utf8]{inputenc}
\usepackage[english,russian]{babel}
\usepackage{indentfirst}%first paragraph indent
\usepackage{cmap}
\usepackage[unicode=true]{hyperref}
\usepackage{graphicx}
\usepackage{amssymb}
\usepackage{amsmath}
\usepackage{srcltx}
\usepackage{textcomp}
\usepackage{floatflt}
\usepackage{wrapfig}
\usepackage{afterpage}
\usepackage{ccaption}
\captiondelim{. }
\usepackage{xspace}
%научные символы и смайлики \smiley \frownie
\usepackage{wasysym}
\usepackage[noae]{Sweave}
\usepackage{underscore}
%переопределение примечания для KOMA-script
%   [отступ до метки]{отступ}{отступ абзаца}
\deffootnote[2.5em]{1.5em}{1em}{\textsuperscript{\thefootnotemark}}
\setkomafont{sectioning}{\bfseries}
\setkomafont{descriptionlabel}{\bfseries}


%backslash
\newcommand{\bs}{\symbol{'134}}
%degree
\newcommand{\grad}{\ensuremath{{}^{\circ}}\xspace}
%R-акроним
\newcommand{\R}{{\sffamily\bfseries R}\xspace}

\title{Типы данных в \R \\ и принципы работы с ними}
\author{\textcopyright{} А.Б.\,Шипунов\thanks{e-mail: dactylorhiza@gmail.com},
     Е.М.\,Балдин\thanks{e-mail: E.M.Baldin@inp.nsk.su}}

\begin{document}
\maketitle

%\tableofcontents


\section{Числовые векторы}
\label{sec:vectors}

Простейший вектор (рост сотрудников):
<<echo=TRUE,print=TRUE>>=
x <- c(174, 162, 188, 192, 165, 168, 172)
@

Структура вектора:
<<echo=TRUE,print=TRUE>>=
str(x)
@ 

Проверка: <<А вектор ли это?>>:
<<echo=TRUE,print=TRUE>>=
is.vector(x)
@ 
 
\section{Факторы}
\label{sec:factors}

Текстовый (character) вектор (пол сотрудников):
<<echo=TRUE,print=TRUE>>=
sex <- c("male", "female", "male", "male", "female", "male", "male")
is.character(sex)
is.vector(sex)
str(sex)
@ 

Содержимое текстового вектора:
<<echo=TRUE,print=TRUE>>=
sex
sex[1]
table(sex)
@ 

Задаём фактор:
<<echo=TRUE,print=TRUE>>=
sex.f <- factor(sex)
sex.f
@ 

График:
<<echo=TRUE,print=TRUE,fig=TRUE>>=
plot(sex.f)
@ 

Что такое фактор:
<<echo=TRUE,print=TRUE>>=
is.factor(sex.f)
is.character(sex.f)
str(sex.f)
@ 

Доступ к данным:
<<echo=TRUE,print=TRUE>>=
sex.f[5:6]
sex.f[6:7]
sex.f[6:7, drop=TRUE]
factor(as.character(sex.f[6:7]))
@ 

Приведение к цифровому виду:
<<echo=TRUE,print=TRUE>>=
as.numeric(sex.f)
@ 

Ещё один вектор (вес сотрудников):
<<echo=TRUE,print=TRUE>>=
w <- c(69, 68, 93, 87, 59, 82, 72)
@ 

Построение графика:
<<echo=TRUE,print=TRUE,fig=TRUE>>=
plot(x, w, pch=as.numeric(sex.f), col=as.numeric(sex.f))
legend("topleft", pch=1:2, col=1:2, legend=levels(sex.f))
@ 

Ещё один текстовый вектор (размер маек сотрудников):
<<echo=TRUE,print=TRUE>>=
m <- c("L", "S", "XL", "XXL", "S", "M", "L")
m.f <- factor(m)
m.f
@ 

Упорядочиваем текстовые данные:
<<echo=TRUE,print=TRUE>>=
m.o <- ordered(m.f, levels=c("S", "M", "L", "XL", "XXL"))
m.o
@ 

\section{Пропущенные данные}
\label{sec:nodata}


Вектор данных с пропущенными значениями (время на сон) :
<<echo=TRUE,print=TRUE>>=
h <- c(8, 10, NA, NA, 8, NA, 8)
h
@ 

Вычисление среднего если есть пропущенные данные:
<<echo=TRUE,print=TRUE>>=
mean(h)
mean(h, na.rm=TRUE)
mean(na.omit(h))
@ 

Замена пропущенных данных на среднее по выборке:

<<echo=TRUE,print=TRUE>>=
h[is.na(h)] <- mean(h, na.rm=TRUE)
h
@ 


\section{Матрицы}
\label{sec:matrix}

Матрицы \(2\times2\):
<<echo=TRUE,print=TRUE>>=
m <- 1:4
m
ma <- matrix(m, ncol=2, byrow=TRUE)
ma
str(ma)
str(m)
mb <- m
mb
attr(mb, "dim") <- c(2,2)
mb
@ 
Матрица \(2\times2\times2\):
<<echo=TRUE,print=TRUE>>=
m3 <- 1:8
dim(m3) <- c(2,2,2)
m3
@ 

\section{Списки}
\label{sec:lists}

Обычный список:
<<echo=TRUE,print=TRUE>>=
l <- list("R", 1:3, TRUE, NA, list("r", 4))
l
@ 

Способы доступа к данным:
<<echo=TRUE,print=TRUE>>=
h[3]
ma[2, 1]
l[1]
str(l[1])
l[[1]]
str(l[[1]])
@

Именование списка:
<<echo=TRUE,print=TRUE>>=
names(l) <- c("first", "second", "third", "fourth", "fifth")
str(l$first)
@ 

Именование векторов и матриц:
<<echo=TRUE,print=TRUE>>=
names(w) <- c("Коля", "Женя", "Петя", "Саша", "Катя", "Вася", "Жора")
w
w["Женя"]
rownames(ma) <- c("a1","a2")
colnames(ma) <- c("b1","b2")
ma
@ 

\section{Таблицы данных}
\label{sec:tables}


Пример таблицы данных:
<<echo=TRUE,print=TRUE>>=
d <- data.frame(weight=w, height=x, size=m.o, sex=sex.f)
d
str(d)
@ 

Доступ к данным таблицы:
<<echo=TRUE,print=TRUE>>=
d$weight
d[[1]]
d[,1]
d[,"weight"]
@ 

Выбор нужных колонок:
<<echo=TRUE,print=TRUE>>=
d[,2:4]
d[,-1]
@ 

Выборка  данных по условию:
<<echo=TRUE,print=TRUE>>=
d$sex=="female"
d[d$sex=="female",]
@ 

Сортировка выборки:
<<echo=TRUE,print=TRUE>>=
d[order(d$sex, d$height), ]
@ 


\section{Векторизованные вычисления}
\label{sec:vectorcount}

Простые операции над векторами:
<<echo=TRUE,print=TRUE>>=
w*1000
@ 

Циклы:
<<echo=TRUE,print=TRUE>>=
for (i in seq_along(w)) {
 w[i] <- w[i] * 1000
}
w
@ 

Операции с матрицами:
<<echo=TRUE,print=TRUE>>=
ma + mb
1:8 + 1:2
@ 

Векторные операции:
<<echo=TRUE,print=TRUE>>=
apply(trees, 2, mean)
sapply(d, as.numeric)
by(d[,1:2], d$sex, mean)
lapply(d, is.factor)
@ 


\end{document}