
\documentclass[russian]{beamer}
\usepackage[T1]{fontenc}
\usepackage[utf8]{inputenc}
\setcounter{secnumdepth}{3}
\setcounter{tocdepth}{3}

\makeatletter

%%%%%%%%%%%%%%%%%%%%%%%%%%%%%% LyX specific LaTeX commands.
\DeclareRobustCommand{\cyrtext}{%
  \fontencoding{T2A}\selectfont\def\encodingdefault{T2A}}
\DeclareRobustCommand{\textcyr}[1]{\leavevmode{\cyrtext #1}}
\AtBeginDocument{\DeclareFontEncoding{T2A}{}{}}


%%%%%%%%%%%%%%%%%%%%%%%%%%%%%% Textclass specific LaTeX commands.
 % this default might be overridden by plain title style
 \newcommand\makebeamertitle{\frame{\maketitle}}%
 \AtBeginDocument{
   \let\origtableofcontents=\tableofcontents
   \def\tableofcontents{\@ifnextchar[{\origtableofcontents}{\gobbletableofcontents}}
   \def\gobbletableofcontents#1{\origtableofcontents}
 }

%%%%%%%%%%%%%%%%%%%%%%%%%%%%%% User specified LaTeX commands.
\mode<presentation>
{
  \usetheme{CambridgeUS}
  % or ...

  \setbeamercovered{transparent}
  % or whatever (possibly just delete it)
}



\usepackage{babel}

\makeatother
%\makebeamertitle{Data type in R  and how to work with them}

\usepackage{babel}
\author{D. Stefanovskiy}
\institute{RANE}
\date{lef,fff}
\begin{document}
\begin{frame}{мммм}
\frametitle{рррррр}
\framesubtitle{р1111ррррр}
\section{Название проекта}
Финасирование покупки новых технологий

\end{frame}

\begin{frame}{Участники}

\section{Участники}
Чешский экспортный банк

Российская компания(потребность во внедрении передовых технологий) 

Структура Garant. (Страховая компания)

\end{frame}

\begin{frame} {Шаг 1}
\section{Алгоритм}
Российская компания заключает контракт на поставку оборудования с
Чешской компанией. Оборудование, его спецификация, а также производитель
определяется самой компанией.

\end{frame}

\begin{frame} {Шаг 2}

Структура Garant предоставляет гарантийные обязательства в чешский
банк, о готовности оплатить поставку оборудования для Российской компании,в
случае если она будет неспособна выполнить условия оплаты по контракту
с SGgroup. На этом шаге учточняются срок налоговых каникул и интересы
сторон

\end{frame}

\begin{frame} {Шаг 3}

На основе гарантийных обязательств и контракта с Российской компанией
SGgroup получает кредитную линию на закупку, поставку и установку
оборудования. 

\end{frame}

\begin{frame} {Шаг 4}

SGgroup заказывает необходимое для Российской компании и оборудование
и технику у производителей ,выставляя в банк фактуры на оплату оборудование
и техники компании производителю.(Фактически производит оплату оборудования
поставщику). 

\end{frame}

\begin{frame} {Шаг 5}

Оборудование поставляется Российской компании ,которая ставит его
себе на баланс, подтверждая поставку актом приема передачи. Привлекается
ГУП Ростек  для консультаций в оформлении документов необходимых
для поставки оборудования и материалов в РФ. 

\end{frame}

\begin{frame} {Шаг 6}

Оборудование поставляется Российской компании, которая ставит его
себе на баланс, подтверждая поставку актом приема передачи. Для консультаций
в оформлении документов, связанных с поставкой оборудования и материалов
в РФ, а также для оптимизации самой поставки привлекается ГУП Ростек
. 

\end{frame}

\begin{frame} {Шаг 7}

После окончания срока каникул по платежам за поставленное оборудование(сроки
оговоренны в контракте на постаку), Российсквя компания начинает выполнять
платежи по погашению суммы долга за поставленное оборудование,согласно
условиям контракта.

\end{frame}

\begin{frame} {Дополнительная информация}
\section {Информация об участниках} 

В случае невозможности осуществления платежей по погашению долговых
обязательств со стороны Российской компании, банк запускает механизм
гарантийных обязательств, требуя выполнения обязательств по погашению
долга от структуры гаранта. 

Не одна из сторон не имеет возможность использования кредитной линии
, как свободных денежных средств, только как ресурсы к оплате оборудования
и услуг связанных с выполнением условий по контракту, и только на
основе фактур выставленных SGgroup в банк, и подтвержденных банком. 

\end{frame}
\end{document}
